%% 
%% Copyright 2007, 2008, 2009 Elsevier Ltd
%% 
%% This file is part of the 'Elsarticle Bundle'.
%% ---------------------------------------------
%% 
%% It may be distributed under the conditions of the LaTeX Project Public
%% License, either version 1.2 of this license or (at your option) any
%% later version.  The latest version of this license is in
%%    http://www.latex-project.org/lppl.txt
%% and version 1.2 or later is part of all distributions of LaTeX
%% version 1999/12/01 or later.
%% 
%% The list of all files belonging to the 'Elsarticle Bundle' is
%% given in the file `manifest.txt'.
%% 

%% Template article for Elsevier's document class `elsarticle'
%% with numbered style bibliographic references
%% SP 2008/03/01

\documentclass[preprint,1p]{elsarticle}
\biboptions{numbers,sort&compress}

%% Use the option review to obtain double line spacing
%% \documentclass[authoryear,preprint,review,12pt]{elsarticle}

%% Use the options 1p,twocolumn; 3p; 3p,twocolumn; 5p; or 5p,twocolumn
%% for a journal layout:
%% \documentclass[final,1p,times]{elsarticle}
%% \documentclass[final,1p,times,twocolumn]{elsarticle}
%% \documentclass[final,3p,times]{elsarticle}
%% \documentclass[final,3p,times,twocolumn]{elsarticle}
%% \documentclass[final,5p,times]{elsarticle}
%% \documentclass[final,5p,times,twocolumn]{elsarticle}

%% For including figures, graphicx.sty has been loaded in
%% elsarticle.cls. If you prefer to use the old commands
%% please give \usepackage{epsfig}

%% The amssymb package provides various useful mathematical symbols
\usepackage{amssymb}
\usepackage{lineno}
\usepackage{hyperref}
%% The amsthm package provides extended theorem environments
%% \usepackage{amsthm}

%% The lineno packages adds line numbers. Start line numbering with
%% \begin{linenumbers}, end it with \end{linenumbers}. Or switch it on
%% for the whole article with \linenumbers.
%% \usepackage{lineno}

\journal{Nucl. Instrum. Meth. A}

\begin{document}
\linenumbers

\begin{frontmatter}

%% Title, authors and addresses

%% use the tnoteref command within \title for footnotes;
%% use the tnotetext command for theassociated footnote;
%% use the fnref command within \author or \address for footnotes;
%% use the fntext command for theassociated footnote;
%% use the corref command within \author for corresponding author footnotes;
%% use the cortext command for theassociated footnote;
%% use the ead command for the email address,
%% and the form \ead[url] for the home page:
%% \title{Title\tnoteref{label1}}
%% \tnotetext[label1]{}
%% \author{Name\corref{cor1}\fnref{label2}}
%% \ead{email address}
%% \ead[url]{home page}
%% \fntext[label2]{}
%% \cortext[cor1]{}
%% \address{Address\fnref{label3}}
%% \fntext[label3]{}

\title{Test Beam Studies Of Silicon Timing for Use in Calorimetry.}

%% use optional labels to link authors explicitly to addresses:
%% \author[label1,label2]{}
%% \address[label1]{}
%% \address[label2]{}

\author[1]{A.~Apresyan}
\author[2]{G.~Bolla}
\author[1]{A.~Bornheim}
\author[3]{H.~Kim}
\author[2]{S.~Los\corref{cor}}
\ead{los@fnal.gov}
\author[1]{C.~Pena}
\author[2]{E.~Ramberg}
\author[2]{A.~Ronzhin}
\author[1]{M.~Spiropulu}
\author[1]{S.~Xie}
\address[1]{California Institute of Technology, Pasadena, CA, USA}
\address[2]{Fermi National Accelerator Laboratory, Batavia, IL, USA}
\address[3]{University of Chicago, Chicago, IL,  USA}
\cortext[cor]{Corresponding author}

\begin{abstract}
%% Text of abstract
The high luminosity upgrade of the Large Hadron Collider (HL-LHC) at CERN is
expected to provide instantaneous luminosities of $5\times 10^{34}$ cm$^{-2}$
s$^{-1}$. The high luminosities expected at the HL-LHC will be accompanied by a
factor of $5$ to $10$ more pileup compared with LHC conditions in $2015$,
further increasing the challenge for particle identification and event
reconstruction. Precision timing allows to extend calorimetric measurements into
such a high density environment by subtracting the energy deposits from pileup
interactions. Calorimeters employing silicon as the active component have
recently become a viable choice for the HL-LHC and future collider experiments
which face very high radiation environments. In this article, we present studies
of basic calorimetric and precision timing measurements using a prototype
composed of tungsten absorber and silicon sensor as the active medium. We show
that for the bulk of electromagnetic showers induced by electrons in the range
of $20$~GeV to $30$~GeV, we can achieve time resolutions better than $25$~ps per
single pad sensor. 

\end{abstract}

\begin{keyword}
%% keywords here, in the form: keyword \sep keyword

%% PACS codes here, in the form: \PACS code \sep code
Silicon \sep Timing \sep Calorimeter
%% MSC codes here, in the form: \MSC code \sep code
%% or \MSC[2008] code \sep code (2000 is the default)

\end{keyword}

\end{frontmatter}

%% \linenumbers

%% main text
\section{Introduction} 

Future colliders, including the high luminosity upgrade of the Large Hadron
Collider (HL-LHC) at CERN, will operate with an order of magnitude higher
instantaneous luminosity compared to what has been achieved at the LHC so far.
With the increased instantaneous luminosity the rate of simultaneous
interactions per bunch crossing (pileup) is projected to reach an average of 140
to 200. The large amount of pileup increases the likelihood of confusion in the
reconstruction of particles from the hard scatter interaction with those
produced in different pileup interactions. The ability to discriminate between
jets produced in the events of interests, especially those associated with the
vector boson fusion processes, and jets produced by pileup interactions will be
degraded. The missing transverse energy resolution will deteriorate, and several
other physics objects performance metrics will suffer.

One way to mitigate the pileup confusion effects, complementary to precision
tracking methods, is to perform a time of arrival measurement associated with a
particular layer of the calorimeter, allowing for a time assignment for 
charged particles and photons. Such a measurement with a precision of about
20-30 ps, when unambiguously associated to the corresponding energy measurement,
will reduce the effective amount of pileup by a factor of 10, given that the
spread in collision time of the pileup interactions at HL-LHC is foreseen to be
approximately 200~ps. The association of the time measurement with the energy
measurement is crucial, and leads to a prototype design that calls for  time
and energy measurements to be performed in the same detector element. Since both
the energy and time measurement are performed in the same detector
element\footnote{If there are no overlapping energy deposits in the same
detector element from multiple particles.}, once
an energy deposit is identified as originating from a pileup interaction, it can
be unambiguously removed from event reconstruction. 

Several alternative options to combine high resolution energy and timing
measurements for calorimetry have been reported in Refs.~\cite{Anderson:2015gha,
MCPFastCaloNIMA, Ronzhin2015288, Ronzhin201552, Brianza2015216}. In this
article, we describe the continuation of this program of study using a
calorimeter prototype employing a 300~$\mu$m thick silicon pad sensor of
$6\times 6$~mm$^2$ size as the active element. Silicon-based calorimeters have
recently become a viable choice for future colliders due to the radiation
hardness of silicon, and the ability to construct highly granular
detectors~\cite{Adloff:2011ha}. An important example is the forward calorimeter
proposed for the CMS Phase 2 Upgrade~\cite{Butler:2020886}. We study the timing
properties of silicon-based calorimetery using a prototype composed of tungsten
absorber and a silicon sensor produced by Hamamatsu~\cite{hamamatsu}. A similar
test was previously conducted at the CERN North Area, with a lead
absorber followed by silicon sensors of 120-320~$\mu$m thickness~\cite{akchurin}.

The paper is organized as follows. General silicon timing properties and bench
test results are described in Section~\ref{sec:siliconpad}. The test beam setup
and experimental apparatus are presented in Section~\ref{sec:tbeam}. The results
of the test beam measurements are presented in Section~\ref{sec:results}.
Sections~\ref{sec:discussion}~and~\ref{sec:conclusion} are devoted to discussion
and conclusion, respectively.

\section{General Properties of Silicon Timing and Bench Test Studies}
\label{sec:siliconpad}

For our measurements, we used a silicon sensor produced by
Hamamatsu~\cite{hamamatsu}. The thickness of the silicon was measured to be ~325
$\mu$m. The transverse size of the sensor is 6x6 mm$^2$. The negative bias
voltage was applied to the p-side of the silicon. The capacitance
of the silicon diode is measured as a function of the bias voltage
and shown in Figure~\ref{fig:SiliconDiode}. We observe that the silicon
is fully depleted above about $120$~V. Timing measurements are expected
to improve with larger bias voltage as the the carrier velocity increases.

The electric diagram of the silicon
diode connections is presented in Figure~\ref{fig:SiliconPad}. Attention was
paid to provide good filtering for bias voltage, to reduce ground loop effects, and
to minimize inductive loop for the signal readout. The timing characteristics
of the signal pulses are dominated primarily by properties of the
silicon sensor rather than the details of the circuit.

\begin{figure}[htbp] 
\centering
\includegraphics[width=0.8\textwidth]{plots/SiliconDiodeCV_v3.pdf} 
\caption{The measured capacitance as a function of the applied bias voltage.} 
\label{fig:SiliconDiode} 
\end{figure} 

The silicon diode was placed inside a light-tight box of thickness $1.5$~cm,
which also provides electromagnetic shielding. The box is made of 0.2~mm steel.
The bias voltage was supplied to the circuitry by a cable with a balun filter,
terminated with an SHV connector. The silicon diode output signal is read out
through an SMA connector electrically grounded to the box. The dark current was
measured at several values of the bias voltage. The maximum value of the dark
current was less than $1.0$~nA at $-500$~V, which is the largest bias voltage
used in the measurements reported in this paper. The silicon box and bench test
setup are presented in Figure~\ref{fig:SiliconPad}. 

\begin{figure}[htbp] 
\centering
\includegraphics[width=0.60\textwidth]{plots/SiliconDiodeDiagram.pdf} 
\includegraphics[width=0.39\textwidth]{plots/SiliconDiodeBox.jpg} 
\caption{The electric diagram for the silicon diode connections (left). External
view of the box with silicon diode, and the bias voltage connection is shown
below it (right).} 
\label{fig:SiliconPad} 
\end{figure} 

The signals from the silicon sensor were amplified by two fast, high-bandwidth
pre-amplifiers connected in series. The first amplifier is an ORTEC VT120C
pre-amplifier, and the second amplifier is a Hamamatsu C5594 amplifier. Using a
pulse-generator, we measured the combined gain of the two amplifiers in series
as a function of the input signal amplitude and found some degree of
non-linearity for typical signals produced by the silicon sensor under study,
and we corrected for them.

%. The measured gain ranged from $200$ for signals with
%input amplitude around $0.15$~mV to $650$ for signals with amplitude around $10$~mV.

\section{Test-beam Setup and Experimental Apparatus }
\label{sec:tbeam}

We performed the test-beam measurements at the Fermilab Test-beam Facility
(FTBF) which provided a proton beam from the Fermilab Main Injector accelerator
at $120$~GeV, and secondary beams composed of electrons, pions, and muons of
energies ranging from $4$~GeV to $32$~GeV. A simple schematic diagram of the
experimental setup is shown in Figure~\ref{fig:BeamSchematicDiagram}. A small
plastic scintillator of transverse dimensions $1.8$~mm$\times 2$~mm is used as a
trigger counter to initiate the read out of the data acquisition (DAQ) system
and to select incident beam particles from a small geometric area, allowing us to center the 
beam particles on the silicon sensor. Next, we place a stack of tungsten absorbers of various thicknesses for
measurements of the longitudinal profile of the electromagnetic shower. The
silicon pad sensor is located within a metal box covered by copper foil, and is
placed immediately downstream of the absorber plates. Finally, a Photek 240
micro-channel plate photomultiplier detector~\cite{Anderson:2015gha,
MCPFastCaloNIMA, Ronzhin2015288,Ronzhin201552} is placed furthest downstream,
and serves to provide a very precise reference timestamp. Its precision was
previously measured to be less than $10$~ps~\cite{Ronzhin2015288}. 
A photograph showing the various
detector components is presented in Figure~\ref{fig:BeamPhotoDiagram}. A
differential Cherenkov counter is located further upstream of our experimental
setup and provides additional particle identification capability. More
details of the experimental setup are described in our previous studies using
the same experimental facility in references~\cite{Anderson:2015gha,
MCPFastCaloNIMA, Ronzhin2015288,Ronzhin201552}.

\begin{figure}[htbp] 
\centering
\includegraphics[width=0.65\textwidth]{plots/BeamSchematicDiagram.pdf} 
\caption{A schematic diagram of the test-beam setup is shown. The $t_0$ and $t_1$ are defined in Section~\ref{sec:results}.} 
\label{fig:BeamSchematicDiagram} 
\end{figure} 

\begin{figure}[htbp] 
\centering
\includegraphics[width=0.65\textwidth]{plots/BeamPhotoDiagram.pdf} 
\caption{Test beam setup.} 
\label{fig:BeamPhotoDiagram} 
\end{figure} 

The DAQ system is based on a CAEN V1742 digitizer board~\cite{CAENDRS}, which
provides digitized waveforms sampled at 5 GS/s. The metal box containing the
silicon sensor was located on a motorized X-Y moving stage allowing us to change
the location of the sensor in the plane transverse to the beam at an accuracy
better than $0.1$~mm. A nominal bias voltage of $500$~V was applied to deplete
the silicon sensor in most of the studies shown below, unless noted otherwise.


\section{Test Beam Measurements and Results} 
\label{sec:results} 

Measurements were performed in 2015, using the primary 120~GeV proton beam, and secondary
beams provided for the FTBF. Secondary beams with energies ranging from 4
GeV/c$^2$ to 32 GeV/c$^2$ were used. Electron purity for those beams ranges
between $70\%$ at the lowest energy to about $10\%$ at the highest energy.
Stacks of tungsten plates with varying thicknesses were placed immediately
upstream of the silicon device in order to measure the response along the
longitudinal direction of the electromagnetic shower. The radiation length of
tungsten is 3.5~mm, and the Moliere radius is 9.3~mm. The tungsten plate size is
sufficient to fully contain the shower in the transverse dimension. Signals from
the silicon sensor and the Photek MCP-PMT are read out and digitized by the CAEN
V1742 digitizer, and example signal waveforms are shown in
Fig.~\ref{fig:pulses}. The signal pulse in the silicon sensor has a rise time of
about $1.5$~ns, and a full pulse width of around $7$~ns. This rise time is
consistent with a time constant of a silicon sensor coupled to a 50~Ohm amplifier.

\begin{figure}[htbp] 
\centering
\includegraphics[width=0.45\textwidth]{plots/ExampleSiliconPadPulse_6X0_16GeV.pdf} 
\includegraphics[width=0.45\textwidth]{plots/ExamplePhotekPulse.pdf} 
\caption{Examples of the signal pulse waveform for the silicon sensor (left) and
the Photek MCP-PMT (right) digitized by CAEN V1742 digitizer board. The bias
voltage applied to the silicon pad sensor is~$500$~V.} 
\label{fig:pulses} 
\end{figure} 

The CAEN digitizer is voltage and time calibrated using the  procedure
described in Ref.~\cite{Kim201467}. The total collected charge for each signal
pulse is computed by integrating a $10$~ns window around the peak of the pulse.
The time for the reference Photek MCP-PMT detector is obtained by fitting the
peak region of the pulse to a Gaussian function and the mean parameter of the
Gaussian is assigned as the timestamp $t_0$. The time for signals from the
silicon sensor is obtained by performing a linear fit to the rising edge of the
pulse and the time at which the pulse reaches 30\% of the maximum amplitude is
assigned as its timestamp $t_1$. We measured the €œelectronic€ time resolution
of the CAEN V1742 digitizer as $\sim$4~ps and neglected its impact on the timing
measurements described below.

Electrons were identified by requiring that the signal amplitude of the gas Cherenkov counter
provided by the FTBF and the Photek detector located further
downstream of the silicon sensor exceed certain thresholds because electromagnetic showers induced by electrons
produce significantly larger signals, while pions produce much smaller signals. After imposing the electron identification requirements the electron purity is between $80\%$ and $90\%$ for all beam
conditions. The purity was determined by comparing the calorimetric measurements with those 
from the Cherenkov detector.

We begin by establishing the signal characteristics of a minimum-ionizing
particle (MIP) using beams of $120$~GeV protons and $8$~GeV electrons with no
absorbers upstream of the silicon pad sensor. To separate MIP signals from
noise, we first collect data events with no beam and random trigger. The charge
distribution for these noise runs is presented in Fig.~\ref{fig:noise}. As expected,
the charge distribution is centered at $0$, and the RMS is about
$2$~fC. 

\begin{figure}[htbp] 
\centering
\includegraphics[width=0.45\textwidth]{plots/NoiseNoBeam_charge.pdf} 
\caption{The distribution of charge integrated in the silicon sensor is shown for data events with no beam and random trigger. } 
\label{fig:noise} 
\end{figure} 

In Figure~\ref{fig:MIP}, we show silicon sensor response to $120$~GeV protons
and $8$~GeV electrons without any absorber. We observe very similar response for
these two cases, and measure peak integrated charge of $4.5$~fC and $5.0$~fC
respectively. The measured signal is corrected for the gain of the amplifiers
used, and hence is the output charge of the silicon sensor. We expect peak charge
of $28,000$ and $31,000$ electron-hole pairs in a $325$ $\mu$m thick silicon
detector for ionizing particles with Lorentz factor $\gamma=120$ (protons) and $16,000$
(electrons)~\cite{Agashe:2014kda}, which is in a good agreement with the measured
values. Having established the absolute scale of the response using single
particles, in our remaining studies we normalize all charge measurements to the
$120$ GeV proton signal, which we refer to in the following as
$\mathrm{Q}_{\mathrm{MIP}}$. 


%In Figure~\ref{fig:MIP}, we show the response of the silicon sensor to the
%proton and electron beams without any absorbers upstream. All triggered events
%were used in these distributions. We observe very similar response for these two
%cases, and measure an integrated charge of $4.5$~fC and $5.0$~fC for the proton
%and electron beams, respectively. The measured charge is corrected for the gain
%of the amplifiers and attenuators used, and hence is the output of the silicon
%sensor. We expect that a MIP traversing a silicon sensor of thickness 325~$\mu$m
%to produce roughly 35,000 electron-hole pairs~\cite{Agashe:2014kda},
%corresponding to a collected charge of about $5$~fC. The difference in
%ionization losses in silicon is expected to be about 10\% between 120~GeV
%protons and 8~GeV electrons~\cite{Agashe:2014kda}. Thus, our measured values are
%in agreement with expectations. Having established the absolute scale of the
%response using MIPs, in our remaining studies we normalize all charge
%measurements to the charge integrated in the silicon sensor for one MIP,
%$\mathrm{Q}_{\mathrm{MIP}}$. 

\begin{figure}[htbp] 
\centering
\includegraphics[width=0.45\textwidth]{plots/Proton_charge.pdf} 
\includegraphics[width=0.45\textwidth]{plots/Electron_0X0_charge.pdf} 
\caption{The distribution of charge integrated in the silicon sensor is shown
for a beam of $120$~GeV protons (left) and $8$~GeV electrons (right) without any
absorber upstream of the silicon sensor. These conditions mimic the response of the silicon sensor to a
minimum-ionizing particle. All triggered events were used in these
distributions.} 
\label{fig:MIP} 
\end{figure} 

We study the response of the silicon sensor to electron beams of various
energies after 6 radiation lengths ($X_0$) of tungsten absorber. The silicon
sensor is expected to be sensitive to the number of secondary electrons produced
within the electromagnetic shower, and therefore its response is expected to
scale up with higher incident electron energy. In
Figure~\ref{fig:ChargeDistributionExample}, we show an example of the integrated
charge distribution measured in the silicon sensor after 6 radiation lengths of
tungsten, for runs with $32$~GeV electrons. We show the mean and RMS of
these distributions as a function of incident electron beam energy in
Figure~\ref{fig:ChargeDistributionExample}. The uncertainties plotted show the RMS of the
charge distribution. Since the electron beam profile and purity varies at different beam energies, we collected between 10 and 50 thousand events for each beam energy, in order to ensure sufficiently large data samples. We observe a fairly linear depedence between the measured
charge and the incident beam energy, for beam energies between $4$~GeV and
$32$~GeV. 

\begin{figure}[htbp] 
\centering
\includegraphics[width=0.49\textwidth]{plots/Electron_6X0_32GeV_chargeMIP.pdf} 
\includegraphics[width=0.49\textwidth]{plots/MIPVsEnergyAt6X0.pdf} 
\caption{ Left: An example of the distribution of integrated charge in the
silicon sensor shown in units of charge measured for a MIP. Right: The
integrated charge in the silicon sensor expressed in units of the charge
measured for a MIP is shown as a function of the electron beam energy. The
uncertainty bands show the RMS of the measured charge distribution. The red line
is the best fit to a linear function..} 
\label{fig:ChargeDistributionExample}
\end{figure}

We also measure the time resolution between the silicon sensor and the Photek
MCP-PMT, by measuring the standard deviation of the gaussian fit to the
distribution of $\Delta t = t_0-t_1$. We observe a systematic dependence of
$\Delta t$ on the total charge measured in the silicon detector, as shown on the
left panel in Figure~\ref{fig:timewalk}. This dependence on the integrated
charge of the amplified signal was reproduced when we connected the output of
the pulse generator to the same amplifiers as used in the measurements. We
perform a correction to $\Delta t$ for each event using the measured charge in
the silicon sensor. This procedure is referred to in the following as
\textit{time calibration}. The correction is obtained from a second degree
polynomial fit to the distribution of the $\Delta t$ versus total charge
collected in the silicon sensor, as shown in Figure~\ref{fig:timewalk}. We
verify that the time calibration flattens the dependence of the time measurement
on the integrated charge, as shown on the right panel of
Figure~\ref{fig:timewalk}, and improves the time resolution measurement by
$30-35$\%. All time resolution measurements in the rest of this study are
performed after time calibration. An example of a corrected $\Delta t$
distribution for $32$~GeV electrons after 6 $X_0$ is shown on the left of
Figure~\ref{fig:MIPVsEnergy}. Other than the electron
identification requirements, no additional selection requirements on the
amplitude of the signal in the silicon sensor were made. The dependence of the
measured time resolution on the beam energy is shown on the right of
Figure~\ref{fig:MIPVsEnergy}. We observe an improvement in the time resolution
as beam energy increases, and achieve a time resolution of $23$~ps for the
$32$~GeV electron beam.

\begin{figure}[htbp] 
\centering
\includegraphics[width=0.49\textwidth]{plots/DeltaT_vs_Charge_Uncorrected.pdf} 
\includegraphics[width=0.5\textwidth]{plots/DeltaT_vs_Charge_Corrected.pdf} 
\caption{ The dependence of $\Delta$t on the integrated charge in the 
silicon sensor is shown on the left. The red curve represents the fit to the
profile plot of the two dimensional distribution, and is used to correct
$\Delta$t for this effect. On the right, we show the corresponding two dimensional
distribution after performing the correction. A 16 GeV electron beam is used, and the silicon sensor is placed
after 6~$X_0$ of tungsten absorber.
} 
\label{fig:timewalk} 
\end{figure} 

\begin{figure}[htbp] 
\centering
\includegraphics[width=0.49\textwidth]{plots/deltaT_32GeV_6X0.pdf} 
\includegraphics[width=0.49\textwidth]{plots/SigmaT_vs_BeamEnergy_lin30Stamp.pdf} 
\caption{Left: The distribution of $\Delta$t between the silicon sensor and the
Photek MCP-PMT. A 32 GeV electron beam is used, and the silicon sensor is placed
after 6~$X_0$ of tungsten absorber. Right: The measured time resolution between
the silicon sensor and the Photek MCP-PMT reference is shown as a function of
the electron beam energy. The silicon sensor is placed after 6~$X_0$ of
tungsten absorber. } 
\label{fig:MIPVsEnergy} 
\end{figure} 

Furthermore, we study the response and time resolution of the silicon sensor
along the longitudinal direction of the shower development. We measure the
integrated charge and the time resolution as a function of the absorber
thickness and present the results in Figure~\ref{fig:MIPVsAbsorberAt8GeV}, for
electron beam energy of 8 GeV. A typical longitudinal shower profile is
observed, consistent with previous studies performed using a secondary emission
calorimeter prototype based on MCP's~\cite{Ronzhin2015288}, as well as
independent studies of silicon-based calorimeter prototypes~\cite{Muhuri201424}.
The RMS of the integrated charge distribution at each absorber thickness is
relatively large, due to the small transverse size of the active element used in
the experiment. We also observe that the time resolution improves as the shower
develops towards its maximum in the longitudinal direction. 

\begin{figure}[htbp] 
\centering
\includegraphics[width=0.49\textwidth]{plots/MIPVsAbsorberAt8GeV.pdf} 
\includegraphics[width=0.5\textwidth]{plots/SigmaT_vs_X0_lin30Stamp.pdf} 
\caption{On the left, the integrated charge in the silicon sensor expressed in
units of the charge measured for MIPs is shown as a function of the absorber (W)
thickness measured in units of radiation lengths ($X_{0}$). The electron beam
energy was 8 GeV. The uncertainty bands show the RMS of the measured charge
distribution. On the right, the time resolution between the silicon sensor and
the Photek MCP-PMT reference is shown as a function of the absorber thickness. } 
\label{fig:MIPVsAbsorberAt8GeV} 
\end{figure} 

Finally, we studied the dependence of the time resolution as a function of the
bias voltage applied to deplete the silicon sensor. The measurements are shown
in Figure~\ref{fig:SigmaT_vs_DV_lin30Stamp} for $16$~GeV electrons after
6~$X_0$ of tungsten absorber. We find that the time resolution
improves as the bias voltage is increased, which is expected on the basis of 
increased velocity of electrons and holes in silicon at larger bias voltage. 

\begin{figure}[htbp] 
\centering
\includegraphics[width=0.8\textwidth]{plots/SigmaT_vs_DV_lin30Stamp.pdf} 
\caption{The time resolution between the silicon sensor and the Photek MCP-PMT 
reference is shown as a function of bias voltage applied on the silicon sensor. The electron beam
energy was 16 GeV, and the silicon sensor is placed
after 6~$X_0$ of tungsten absorber.} 
\label{fig:SigmaT_vs_DV_lin30Stamp} 
\end{figure} 

\section{Discussion} 
\label{sec:discussion} 

From Figures~\ref{fig:noise}~and~\ref{fig:MIP}, we observe that the noise of the
prototype system is sufficiently low to extract signals from MIPs. Comparing the
RMS of the noise distribution with the mean of the MIP signal, we find a
signal-to-noise ratio around $2$ to $2.5$. A rough estimate from
Figure~\ref{fig:MIP} demonstrates that the efficiency to detect $120$~GeV
protons and $8$~GeV electrons with no absorber present is larger than $80\%$.
Based on the measurements for MIPs, we derive signal distributions for
electromagnetic showers normalized to MIP response, and observe a relatively
linear response to the electron beam energy in the range from 4~GeV to 32~GeV
after 6 $X_0$ of tungsten absorber, as shown in Figure~\ref{fig:MIPVsEnergy}. We also
measure a longitudinal shower profile in Figure~\ref{fig:MIPVsAbsorberAt8GeV}
that is consistent with similar past measurements.

%With regards to the timing measurements, we begin with some general considerations.
%An electric field applied to silicon results in a built-in junction voltage
%($\sim 0.6$~V) which is typical of silicon diodes. The high electric field in silicon
%leads to total depletion as all free charge carriers are removed. 
%When charged particles pass through the totally depleted region in silicon, 
%it ionizes atoms and produces electron-hole pairs which serve as charge carriers.
%The electrons are collected on positive electrode and the holes are collected
%on the negative electrode. The time and jitter associated with relativistic particles 
%traversing through the silicon material can be neglected as it takes 
%less than $1$~ps for relativistic particles to pass through the full $300$~$\mu$m
%of silicon material. At high electric field (more than $10^{5}$~V/cm), the mobility 
%of carriers attain a constant drift velocity of $1\mu$m$/10$~ps in the silicon 
%material~\cite{Ronzhin2015,spieler2005semiconductor}. The time required to collect all ionization electrons
%produced by a charged particle traversing the $300\mu$m is about $3$~ns.
%The electrons produced closer to the positive electron are collected first and are
%most relevant for the time stamp assignment. The time needed to collect the electrons
%produced in the last $3\mu$m of silicon is about $30$~ps, and roughly sets the scale
%of the time measurement.

Our results show that the time stamp associated with electromagnetic showers
induced by electrons with energy between $20$~GeV and $30$~GeV can be measured
with a precision better than $25$~ps. Results of the  measurements reported
in Ref.~\cite{akchurin} showed that a time resolution below $50$~ps could be
achieved for signals larger than 10 equivalent MIPs. We find that the time
response of the amplifier needs to be well calibrated in order to achieve this
result. Subtracting $13$~ps for the resolution of the reference Photek MCP-PMT
detector measured in showers~\cite{Ronzhin2015288} yields a precision close to
$20$~ps. Moreover, we observe an improvement of the time resolution with the
energy of the electron, and more generally with an increase in the signal
amplitude. These measurements demonstrate that a calorimeter based on silicon
sensors as the active medium can achieve intrinsic time resolution at the
$20$~ps level, as long as noise is kept under control. Time jitter arising from
intrinsic properties of the silicon sensor is demonstrated to be well below the
$20$~ps level.

\section{Conclusion}
\label{sec:conclusion} 

The best time resolution of $23$~ps for a silicon sensor was achieved with a $32$~GeV beam 
and with the silicon sensor placed after 6 radiation
lengths of tungsten absorber. Based on our calibration data for the response of the
silicon sensor to MIPs, this measurement corresponds roughly to 
an average of $54$ secondary particles registered from the electromagnetic shower. 
We observe a roughly linearly increasing response as the energy of the electron beam
is increased, and we observe a longitudinal shower profile consistent with similar 
past measurements. This result yields further encouragement to use silicon for 
active layers in calorimeters, as is planned for example for the CMS Phase 2 
upgrade~\cite{Butler:2020886}, and explicitly demonstrates the opportunity 
to use silicon for timing measurements in future calorimeters. To continue, we plan to 
extend our studies to more realistic prototypes covering larger transverse and longitudinal 
regions of the electromagnetic shower and using multiple channels. 


\section{Acknowledgements} Operated by Fermi Research Alliance, LLC under
Contract No. DE-AC02-07CH11359 with the United States Department of Energy.
Supported by funding from California Institute of Technology High Energy Physics
under Contract DE-SC0011925 with the United States Department of Energy. We
thank the FTBF personnel for very good beam conditions during our test beam
time. We also appreciate the technical support of the Fermilab SiDet department
for the production of high quality silicon samples. 

%% The Appendices part is started with the command \appendix;
%% appendix sections are then done as normal sections
%% \appendix

%% \section{}
%% \label{}

%% If you have bibdatabase file and want bibtex to generate the
%% bibitems, please use
%%
%%  \bibliographystyle{elsarticle-num} 
%%  \bibliography{<your bibdatabase>}

%% else use the following coding to input the bibitems directly in the
%% TeX file.

\bibliography{SiliconCalorimeter}{}
\bibliographystyle{ieeetr} 

%\begin{thebibliography}{00}

%% \bibitem{label}
%% Text of bibliographic item

%\bibitem{}

%\end{thebibliography}

\end{document}